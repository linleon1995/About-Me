%-------------------------------------------------------------------------------
%	SECTION TITLE
%-------------------------------------------------------------------------------
\cvsection{Projects}


%-------------------------------------------------------------------------------
%	CONTENT
%-------------------------------------------------------------------------------
\begin{cventries}


%---------------------------------------------------------
  \cventrynotitle
  {
    \begin{cvitems} % Description(s) of tasks/responsibilities
      \linkitem{Quantitative Trading and Backtesting System}{https://github.com/linleon1995/quant}{Kafka, Reinforcement Learning, Time-series database} 
      \begin{itemize}
          % \item {Motivation: A partner, Microsoft, developed a Video Analysis service that is hosted on the cloud and comes with high costs. This project aimed to explore the feasibility of improving performance and reducing costs by implementing an alternative solution.}
          \item {Developed a quantitative trading and backtesting system using Apchie Kafka and a time-series database, supporting real-time data, multiple strategies, and data sources; integrated reinforcement learning to optimize trading decisions.}
      \end{itemize}
      % \item {Spotify Content Discovery}
      % \begin{itemize}
      %   % \item {Motivation: Regularly listening to podcasts but facing challenges in searching for specific episodes or content due to functional limitations.}
      %     \item {Responding to limitations in podcast content search functionality, built a comprehensive solution that collects audio via Spotify API and converts it to text using speech2text models.}
      %     \item {This textual data can then powers an AI enabled chatbot that helps users discover podcast content and strengthens the search functionality of Spotify.}
      % \end{itemize}
      \linkitem{Event Assistant System}{https://github.com/linleon1995/Event-Assistant}{LangChain, Llama.cpp, RAG} 
      \begin{itemize}
          % \item {Motivation: Users of the EHS system portal often provided feedback about the cumbersome process of schedule setup and difficulties in event searches.}
          \item {Built an event assistant system leveraging LangChain, Llama.cpp, and a Retrieval-Augmented Generation (RAG) architecture to enable natural language control over multiple APIs.}
      \end{itemize}
    \end{cvitems}
  }

%---------------------------------------------------------
\end{cventries}

%-------------------------------------------------------------------------------
%	SECTION TITLE
%-------------------------------------------------------------------------------
\cvsection{Projects}


%-------------------------------------------------------------------------------
%	CONTENT
%-------------------------------------------------------------------------------
\begin{cventries}


%---------------------------------------------------------
  \cventrynotitle
    {
      \begin{cvitems} % Description(s) of tasks/responsibilities
        \item {LLM based Video Analysis}
        \begin{itemize}
            \item {動機:合作夥伴Microsoft開發了Video Analysis服務。但是架設於雲端且收費高昂。希望透過實作了解性能以及成本節省的可行性}
            \item {使用LongLLM模型進行視訊分析,並且透過Huggingface部署成簡易的線上服務。}
        \end{itemize}
        \item {Spotify Chat}
        \begin{itemize}
            \item {動機:平時有收聽Podcast的習慣,但是因為功能上的限制難以搜尋特定集數或是內容。}
            \item {使用Spotify API收集語音並使用XX模型轉換為文字,自行架設部署Chat與Semantic Search服務,並且透過問答尋找相關內容。}
        \end{itemize}
        \item {Event Agent (對話式班表設定 / 事件語意搜尋)}
        \begin{itemize}
            \item {動機:EHS系統portal時常有使用者反饋,表示設定班表的繁瑣以及事件搜尋的困難}
            \item {自行架設AI Agent 服務,結合Prompt、RAG資訊為實際Action,並透過API Gateway實現班表的快速設定。}
        \end{itemize}
      \end{cvitems}
    }
    
    % % Custom command for project title
    % \newcommand{\projecttitle}[1]{\textbf{#1}}

    % % Custom command for skill with color
    % \newcommand{\skill}[2]{\textcolor{#1}{#2}} % Usage: \skill{color}{text}

    % % Tabular for project and skills
    % \noindent
    % \begin{tabular}{@{}l@{\hspace{1em}}l@{\hspace{1em}}l@{\hspace{1em}}l@{}}
    %     \projecttitle{Project1} & \vert & \skill{gray}{Skill 1} & \vert & \skill{gray}{Skill 2} & \vert & \skill{gray}{Skill 3} \\
    % \end{tabular}

    % \vspace{0.5em} % Add some vertical spacing

    % \begin{itemize}
    %     \item Item 1
    %     \item Item 2
    % \end{itemize}

%---------------------------------------------------------
\end{cventries}

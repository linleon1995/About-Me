%-------------------------------------------------------------------------------
%	SECTION TITLE
%-------------------------------------------------------------------------------
\cvsection{Work Experience}


%-------------------------------------------------------------------------------
%	CONTENT
%-------------------------------------------------------------------------------
\begin{cventries}


%---------------------------------------------------------
  \cventrynew
    {Machine Learning Engineer} % Job title
    {ASUS AIoT} % Organization
    % {Taipei, Taiwan} % Location
    {Jul. 2021 - Dec. 2024} % Date(s)
    {
        \begin{cvitems} % Description(s) of tasks/responsibilities
          \linkitem{EHS General Mission Framework}{}{System Design}
          \begin{itemize}
              \item Led the refactoring of the EHS Mission Pipeline by redesigning the system architecture and data flow, strengthening role responsibilities, 
              and defining clear interaction relationships between roles.
              \item Assisted in designing the Event Engine’s alert system by establishing clear boundaries and consistent representation for event.
          \end{itemize}
          \linkitem{Mission Inference Optimization}{}{Pytorch | TensorRT}
          \begin{itemize}
              \item By utilizing TensorRT, reimplementing preprocessing and postprocessing with PyTorch, and optimizing model inference tasks through multithreading, 
              the maximum number of model inferences was increased under the condition of maximum tolerable latency. On the same hardware (NVIDIA T4), 
              the number of supported cameras increased from X to Y, representing an improvement of Z%.
          \end{itemize}
          \linkitem{EHS Scheduler}{}{Python | Kubernetes | MongoDB}
          \begin{itemize}
              \item Collaborated on developing the EHS Scheduler for managing task scheduling, 
              responding to inference service deployment requests from the EHS Portal, and handling Kubernetes resource management and scheduling.
          \end{itemize}
          \linkitem{AI Agent Service}{}{FastAPI | Azure AI | Pydantic}
          \begin{itemize}
              \item Developed a unified API interface using FastAPI, integrating third-party AI services such as Azure AI and Copilot, 
              and validated the API interface with Pydantic.
          \end{itemize}
          \linkitem{Machine Learning Algorithm Development}{}{PyTorch | Detection | Pose Estimation}
          \begin{itemize}
              \item Designed and trained a new version of the Protective Personal Equipment (PPE) detection algorithm. 
              Achieved a 34.22\% improvement in accuracy, increasing mAP from 0.301 to 0.404 on the test dataset.
          \end{itemize}
          \linkitem{Snoring Detection}{}{C++ | PyTorch | TFLite}
          \begin{itemize}
              \item Collaborated with ASUS mobile device departments to develop a snoring detection algorithm with an accuracy of 94.12\%. 
              Exported the model to TFLite format and implemented spectrogram transformation in C++. 
              Fine-tuned pretrained models to significantly enhance detection performance in high-noise environments.
          \end{itemize}
          \linkitem{3D Lung Nodule Segmentation}{}{PyTorch | 3D Slicer}
          \begin{itemize}
              \item Developed lung nodule segmentation and classification algorithms in collaboration with a major hospital in Tainan. Achieved an average segmentation DSC of 0.7481 and classification accuracy of 91.84\% on hospital test datasets. 
              Additionally, developed a 3D Slicer plugin to assist the hospital in model inference and visualization of lung nodule predictions.
          \end{itemize}
          \linkitem{EHS Demo Tool}{}{Streamlit}
          \begin{itemize}
              \item Assisted in developing a lightweight frontend interface using Streamlit to quickly showcase EHS service applications in offline scenarios.
          \end{itemize}
        \end{cvitems}
    }
  

%---------------------------------------------------------
\end{cventries}

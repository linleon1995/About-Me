%-------------------------------------------------------------------------------
%	SECTION TITLE
%-------------------------------------------------------------------------------
\cvsection{Work Experience}


%-------------------------------------------------------------------------------
%	CONTENT
%-------------------------------------------------------------------------------
\begin{cventries}

%---------------------------------------------------------
\cventryexp
{Machine Learning Engineer} % Job title
{ASUS AIoT} % Organization
% {Taipei, Taiwan} % Location
{Jul. 2021 - Jul. 2025} % Date(s)
{https://iot.asus.com/}
{
    \begin{cvitems} % Description(s) of tasks/responsibilities
        \linkitem{Mission Inference Optimization}{}{Pytorch, TensorRT}
        \begin{itemize}
            \item Optimized the inference pipeline by leveraging TensorRT, adopting a 
            dynamic batching strategy, switching to asynchronous inference, and offloading 
            computations to the GPU using custom CUDA kernels to maximize throughput. 
            \item As a result, the number of supported real-time video streams per 
            Kubernetes node increased from 8 to \textbf{14} on NVIDIA T4 (\textbf{75\%} improvement), potentially 
            saving clients over \textbf{\$30,000 USD} annually in infrastructure costs.
        \end{itemize}
        \linkitem{EHS General Mission Framework}{}{Python, System Design}
        \begin{itemize}
            \item Built a flexible and extensible CCTV AI inference system 
            supporting multiple AI capabilities—including object detection, 
            action recognition, and other skills—that can be easily integrated 
            and executed through the unified framework.
            \item Closely collaborated with a team of 3 colleagues to enhance the 
            EHS mission pipeline, focused on effective teamwork and clear coordination of 
            responsibilities. Responsible for progress tracking and contributed to hands-on 
            development.
        \end{itemize}
        \linkitem{EHS Scheduler}{}{Python, Kubernetes, MongoDB}
        \begin{itemize}
            \item Contributed to the development of the EHS Scheduler to handle mission scheduling requests from the EHS Portal, 
            orchestrating the full lifecycle of CCTV AI task pods on Kubernetes.
        \end{itemize}
        \linkitem{AI Agent Service}{}{Azure AI, Pydantic}
        \begin{itemize}
            \item Designed and implemented a unified API service to register and 
            forward AI task requests from the EHS Portal, integrating third-party 
            AI services such as Azure AI and Copilot, with robust request validation 
            handled through Pydantic.
        \end{itemize}
        \linkitem{Machine Learning Algorithm Development}{}{PyTorch, Detection, Pose Estimation}
        \begin{itemize}
            \item Designed and trained a new version of the Protective Personal Equipment (PPE) detection algorithm to help ensure workplace safety in TSMC factory. 
            Achieved a 34.22\% improvement in accuracy, increasing mAP from 0.301 to 0.404.
        \end{itemize}
        \linkitem{Snoring Detection}{}{C++, PyTorch, TFLite}
        \begin{itemize}
            \item Developed a snoring detection algorithm using PyTorch, with spectrogram transformation implemented in C++, achieving 
            94.12\% accuracy on the testing dataset. The model was exported to TFLite format for deployment on mobile devices.
        \end{itemize}
        \linkitem{3D Lung Nodule Segmentation}{}{PyTorch, 3D MaskRCNN}
        \begin{itemize}
            \item Developed lung nodule segmentation and classification algorithms by using 3D MaskRCNN. Achieved an average segmentation DSC of 0.7481 and classification accuracy of 91.84\% on hospital test datasets. 
            % \item Developed a 3D Slicer plugin to facilitate model inference and enable visualization of lung nodule predictions, enhancing user accessibility and interpretability.
        \end{itemize}
    %   \linkitem{EHS Demo Tool}{}{Streamlit}
    %   \begin{itemize}
    %       \item Assisted in developing a lightweight frontend interface using Streamlit to quickly showcase EHS service applications in offline scenarios.
    %   \end{itemize}
    \end{cvitems}
}

%---------------------------------------------------------
\end{cventries}

%-------------------------------------------------------------------------------
%	SECTION TITLE
%-------------------------------------------------------------------------------
\cvsection{Work Experience}


%-------------------------------------------------------------------------------
%	CONTENT
%-------------------------------------------------------------------------------
\begin{cventries}


%---------------------------------------------------------
  \cventrynew
    {Machine Learning Engineer} % Job title
    {ASUS AIoT} % Organization
    % {Taipei, Taiwan} % Location
    {Jul. 2021 - Dec. 2024} % Date(s)
    {
      \begin{cvitems} % Description(s) of tasks/responsibilities
        \linkitem{EHS General Mission Framework}{}{System Design}
        \begin{itemize}
            % \item 主導EHS Mission Pipeline的重構工作。重新設計了系統流程與資料結構。協助協作開發視訊串流推論服務,設計架構支持大規模視訊推論工作負載。
            \item 主導EHS Mission Pipeline的重構工作。重新設計了系統架構與資料流,並且強化角色職責以及確立角色間互動關係。
            \item 協助Event Engine設計事件告警系統,制定清晰的事件工作邊界與制訂一致的表達形式。
        \end{itemize}
        \linkitem{Mission Inference Optimization}{}{Pytorch | TensorRT}
        \begin{itemize}
            \item 透過使用TensorRT 模型、Pytoch重新實作推論前後處理、以及使用多執行序來優化模型推論工作,並在最大容忍延遲的條件下增加模型的最大推論數量,在相同硬體條件 (NVIDIA T4)。可支援攝影機由X路提升到Y路,提升幅度為Z\%
        \end{itemize}
        \linkitem{EHS Scheduler}{}{Python | Kubernetes | MongoDB}
        \begin{itemize}
            \item 協作EHS Scheduler用於管理任務排程,負責響應來自EHS Portal的推論服務部署並進行Kubernetes資源的管理與調度。
        \end{itemize}
        \linkitem{AI Agent Service}{}{FastAPI | Azure AI | Pydantic}
        \begin{itemize}
            \item 使用FastAPI實作統一的API介面。整合Azure AI、Copilot 等第三方 AI 服務,並使用Pydantic進行API Interface之驗證。
        \end{itemize}
        \item{Machine Learning Algorithm Develope}{}{PyTorch | Detection | Pose Estimation}
        \begin{itemize}
            \item Protective personal equipment (PPE): 協助設計PPE演算法。並訓練新版本PPE模型,在測試資料集上mAP由 0.301 提升到 0.404,準確度提升 34.22\%
            % \item 設計稀疏事件推論架構: 透過稀疏事件推論架構,提升視訊串流推論效能,預估能夠帶來至少X\%的成本節省。
        \end{itemize}
        % \linkitem{Protective personal equipment (PPE)}{}{Pytorch}
        % \begin{itemize}
            % \item 協助設計PPE演算法。並訓練新版本PPE模型,在測試資料集上mAP由 0.301 提升到 0.404,準確度提升 34.22\%
            % \item 設計稀疏事件推論架構: 透過稀疏事件推論架構,提升視訊串流推論效能,預估能夠帶來至少X\%的成本節省。
        % \end{itemize}
        % \linkitem {MLOps pipeline Develope}{}{}
        % \begin{itemize}
        %     \item EHS Training Service: 使用Airflow、Pytorch建立統一訓練框架。提供簡易、一致的訓練介面。
        % \end{itemize}
        \linkitem{Snoring Detection}{}{C++ | PyTorch | TFLite}
        \begin{itemize}
            \item 與 ASUS 移動裝置部門合作,開發鼾聲偵測演算法,準確率高達 94.12 \%。負責將模型以 TensorFlow 導出為 TFLite 格式,並以 C++ 實現時譜轉換功能。同時,透過微調預訓練模型,大幅提升演算法在高噪音環境下的偵測性能。
        \end{itemize}
        \linkitem{3D Lung Nodule Segmentation}{}{PyTorch | 3D Slicer}
        \begin{itemize}
            \item 與台南大型醫院合作開發肺結節分割與分類演算法,在醫院測試集上平均分割 DSC 達 0.7481,分類準確率達 91.84\%,並開發 3D Slicer 插件協助醫院進行模型推論與肺結節預測可視化。
        \end{itemize}
        \linkitem{EHS Demo Tool}{}{Streamlit}
        \begin{itemize}
            \item 協助使用Streamlit開發輕量化的前端介面,用於在離線狀況快速展示EHS服務應用概念。
        \end{itemize}
      \end{cvitems}
    }


%---------------------------------------------------------
\end{cventries}

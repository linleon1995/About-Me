%-------------------------------------------------------------------------------
%	SECTION TITLE
%-------------------------------------------------------------------------------
\cvsection{Work Experience}


%-------------------------------------------------------------------------------
%	CONTENT
%-------------------------------------------------------------------------------
\begin{cventries}


%---------------------------------------------------------
  \cventry
    {Machine Learning Engineer} % Job title
    {ASUS AIoT} % Organization
    {Taipei, Taiwan} % Location
    {Jul. 2021 - Jul. 2024} % Date(s)
    {
      \begin{cvitems} % Description(s) of tasks/responsibilities
        \item {Environment, health and safety (EHS) Cloud Service}
        \begin{itemize}
            \item 協助開發Scheduler用於定期部署視訊串流推論服務
            \item 協助開發視訊串流推論框架
            \item 協助進行Azure AI Service 整合工作
        \end{itemize}
        \item {Computer Vision Algorithm Develope}
        \begin{itemize}
            \item 協助開發PPE演算法,證明
            \item 協助開發Motion Detection算法並透過減少無效推論節省客戶大量成本
            \item MMI
            \item Unsupervised Domain Adaptation
        \end{itemize}
        \item {MLOps pipeline Develope}
        \begin{itemize}
            \item 透過整合Airflow, Mlflow, DVC, CVAT, Git等工具,建立完整的模型訓練流程。並節省團隊大量模型開發時間。
        \end{itemize}
        \item Snoring Detection on Mobile Device
        \begin{itemize}
            \item 與 ASUS 移動裝置部門合作,開發準確率達 94.12\% 的鼾聲偵測演算法,並協助實作使用 TensorFlow 導出 TFLite 模型並以 C++ 實作頻譜轉換,透過微調預訓練模型增強在高噪音環境下的偵測表現。
        \end{itemize}
        \item 3D Lung Nodule Segmentation
        \begin{itemize}
            \item 與台南大型醫院合作開發肺結節分割與分類演算法,在醫院測試集上平均分割 DSC 達 0.7481,分類準確率達 91.84\%,並開發 3D Slicer 插件協助醫院進行模型推論與肺結節預測可視化。
        \end{itemize}
      \end{cvitems}
    }


%---------------------------------------------------------
\end{cventries}

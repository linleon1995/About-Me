%-------------------------------------------------------------------------------
%	SECTION TITLE
%-------------------------------------------------------------------------------
\cvsection{Work Experience}


%-------------------------------------------------------------------------------
%	CONTENT
%-------------------------------------------------------------------------------
\begin{cventries}


%---------------------------------------------------------
  \cventry
    {Machine Learning Engineer} % Job title
    {ASUS AIoT} % Organization
    {Taipei, Taiwan} % Location
    {Jul. 2021 - Jul. 2024} % Date(s)
    {
      \begin{cvitems} % Description(s) of tasks/responsibilities
        \item {Environment, health and safety (EHS) Cloud Service}
        \begin{itemize}
            \item 協助開發Scheduler用於定期部署Kubernetes Pod
        \end{itemize}
        \item Snoring Detection on Mobile Device \hfill Jun. 2022 - Jul. 2023
        \begin{itemize}
            \item 與ASUS移動裝置相關部門合作並進行鼾聲偵測的演算法開發。準確率可達94.12 \%
            \item 使用Tensorflow導出TFLite模型並實作頻譜轉換的C++實現。
            \item 透過調整預訓練模型,增強模型在較高環境噪音下的鼾聲偵測表現。
        \end{itemize}
        \item 3D Lung Nodule Segmentation \hfill Aug. 2021 - Jul. 2022
        \begin{itemize}
            \item 與台南市大型醫院合作三維肺結節分割專案。
            \item 開發肺結節分割及分類演算法。在醫院提供的測試集中,平均分割DSC可達到 0.7481,良惡性腫瘤分類準確率可以達到 0.9184
            \item 使用Slicer開發應用程式,協助醫院人員進行模型推論,並進行肺結節的標記工作。
        \end{itemize}
      \end{cvitems}
    }


%---------------------------------------------------------
\end{cventries}

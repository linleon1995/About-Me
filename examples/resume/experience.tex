%-------------------------------------------------------------------------------
%	SECTION TITLE
%-------------------------------------------------------------------------------
\cvsection{Work Experience}


%-------------------------------------------------------------------------------
%	CONTENT
%-------------------------------------------------------------------------------
\begin{cventries}


%---------------------------------------------------------
  \cventrynew
    {Machine Learning Engineer} % Job title
    {ASUS AIoT} % Organization
    % {Taipei, Taiwan} % Location
    {Jul. 2021 - Dec. 2024} % Date(s)
    {
      \begin{cvitems} % Description(s) of tasks/responsibilities
        \item {Environment, health and safety (EHS) Cloud Service}
        \begin{itemize}
            % \item 協作開發視訊串流推論服務,包含Inference Service用於視訊串流推論、Scheduler用於定期部署推論服務、Event Engine用於確認事件。
            % \item 優化模型推論工作
            % \item 協作開發AI Agent Service,用於整合第三方AI服務,並提供統一介面。
            \item Video Stream Inference Service: 協作開發視訊串流推論服務,設計架構支持大規模視訊推論工作負載。
            \item Scheduler: 協作開發任務排程工具,用於定期部署推論服務。
            \item Event Engine: 協助設計事件告警系統,界定服務工作邊界與制訂一致的事件表達形式。
            \item Inference Optimization: 使用TensorRT優化模型推論工作,提升推論性能。
            \item AI Agent Service: 與團隊協作整合第三方 AI 服務,提供統一 API 界面。
        \end{itemize}
        \item {Demo Tool}
        \begin{itemize}
            \item 協助使用Streamlit開發輕量化的Demo工具,用於展示EHS服務相關成果。
        \end{itemize}
        \item {Machine Learning Algorithm Develope}
        \begin{itemize}
            \item Protective personal equipment (PPE): 協助設計PPE演算法。並訓練新版本PPE模型,在測試資料集上mAP由X提升到Y,準確度提升Z\%
            % \item 設計稀疏事件推論架構: 透過稀疏事件推論架構,提升視訊串流推論效能,預估能夠帶來至少X\%的成本節省。
        \end{itemize}
        % \item {MLOps pipeline Develope}
        % \begin{itemize}
        %     \item EHS Training Service: 使用Airflow、Pytorch建立統一訓練框架。提供簡易、一致的訓練介面。
        % \end{itemize}
        \item \textbf{\textcolor{blue}{Medical AI}} \textbf{\textcolor{gray}{| C++ | TensorFlow | PyTorch}}
        % \item {Medical AI}
        \begin{itemize}
            \item Snoring Detection: 與 ASUS 移動裝置部門合作,開發鼾聲偵測演算法,準確率高達 94.12 \%。負責將模型以 TensorFlow 導出為 TFLite 格式,並以 C++ 實現頻譜轉換功能。同時,透過微調預訓練模型,大幅提升演算法在高噪音環境下的偵測性能。
        \end{itemize}
        \begin{itemize}
            \item 3D Lung Nodule Segmentation: 與台南大型醫院合作開發肺結節分割與分類演算法,在醫院測試集上平均分割 DSC 達 0.7481,分類準確率達 91.84\%,並開發 3D Slicer 插件協助醫院進行模型推論與肺結節預測可視化。
        \end{itemize}
      \end{cvitems}
    }


%---------------------------------------------------------
\end{cventries}
